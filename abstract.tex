This work presents the evolution of research repositories, from collections of peer-reviewed outputs, such as journal articles or monographs, to collections holding a wide array of materials, including data sets, preprints, scientific software or protocols, based on the new realities of the scientific research endeavour. It also discusses a number of novel practical solutions to repository issues such as licencing, bibliographic record modelling, dissemination and linking, or record management and migration by employing new technologies such as linked data, blockchain, and extract, transform and load frameworks.\\\par


Această lucrare prezintă evoluția depozitelor de materiale provenite din activitățile de cercetare, de la biblioteci ce stochează publicații verificate, precum jurnale sau monografii, la colecții digitale vaste ce includ seturi de date, \emph{preprint}-uri, software științific sau protocoale, în acord cu noile realități ale domeniului cercetării. Sunt discutate soluții practice la probleme existente în acest tip de aplicație, cum ar fi licențierea materialelor, modelarea, diseminarea și corelarea informațiilor bibliografice sau administrarea și migrarea lor, soluții ce fac uz de tehnologii noi precum \emph{linked data}, \emph{blockchain} sau sisteme de extragere, transformare și încărcare.  