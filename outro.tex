This work analysed research repositories, by first presenting the state of the general scientific endeavour, namely its challenges (e.g., the reproducbility crisis) and new workflows (e.g., \gls{oa}, \gls{ntro}). Having \glspl{ir} as a starting point, it outlined how this type of digital libraries need to evolve in order to adapt to the new realities, either by implementing new functionality, or by adopting a distributed model in which distinct systems interoperate, in order to achieve the common vision of the scientific community, of research repositories able to preserve, disseminate and link any type  output, no matter its structural or semantic characteristics.

The new generation of repositories is defined by the implementation of complex and novel workflows and thus requires both ingenuity but also inspiration from other domains in order to develop the requisite functionality. Software engineering is one such domain and in this work three original contributions employing concepts from this discipline are presented:
\begin{enumerate}
    \item A licencing and reuse monitoring system implemented using blockchains and smart contracts.
    \item A bibliographical data model, characterised by flexibility and facilitation of interoperability, based on \gls{rdf}.
    \item An \gls{etl} pipeline for migrating bibliographical records across repositories while providing curation opportunities to administrators.
\end{enumerate}

These are only a few of the building blocks that new repository solutions will require; apart from the knowledge transfer from other domains, they also conjecture a possibly distributed architecture for repositories, similar to the design of \emph{microservices}. Such a structure could ensure that each functionality provided by a repository is handled by the party best suited for it, both at the technological and human resource levels.

Nevertheless, this work cannot be in any way prescriptive on the design of the next generation of research repositories. For example, the recent \gls{covid} epidemic has made the case for the fast publication, dissemination and review of science\cite{cochran}, which might come in direct contradiction with the more intricate workflows presented in the previous chapters. At the same time, the fact that current \glspl{ir} need to evolve in order to support \glspl{ntro} has become an accepted truth and thus, various options for either transforming existing solutions or building new ones need to be considered.

To conclude, it is of utmost importance that research on repositories is carried on and that this research is fully connected to the procedural and social realities of the scientific enterprise. The reproducbility crisis and the rising impact of preprints demonstrated that repositories can no longer act as an auxiliary component of the research life cycle but as an integral part of it, providing a stage for the dissemination of science and a platform upon which both \emph{humans} and \emph{machines} can discover new ways of interacting with content, as envisioned by the \gls{fair} principles. As a subset of libraries, repositories play a key role in preserving an integral part of a civilisation's heritage, scientific research, and thus need a high degree of attention when considering the technological means for properly achieveing their functions.